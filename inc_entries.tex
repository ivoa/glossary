% Glossary of typical IVOA terms.
% This is an file expected to be included into a proper tex document. 
%
% In order to make the listing right, you will have to define a
% newcommand {\glossitem} in your document preamble. It should look
% similar to this: 
%
%\newcommand{\glossitem}[2]{
%\item {#1} : \color{black} {#2}
%}


\begin{description}

\glossitem{ADQL}{(Astronomical Data Query Language) is the language used by the International Virtual Observatory Alliance (IVOA) to represent astronomy queries posted to VO services. ADQL is based on the Structured Query Language (SQL), especially on SQL 92. The VO has a number of tabular data sets and many of them are stored in relational databases, making SQL a convenient access means. A subset of the SQL grammar has been extended to support queries that are specific to astronomy.}
\glossitem{Agent}{Software that acts or works on behalf of a user.}
\glossitem{AJAX}{(Asynchronous Javascript + XML) A framework for adding dynamic interactions within web pages.}
\glossitem{Aladin}{An interactive tool that allows the user to visualize digitized images and catalogs from many sources.}
\glossitem{Ant}{A Java -based software build tool, similar to Unix ""make"".}
\glossitem{API}{(Application Programming Interface) The documentation of the interface to a software library or tool.}
\glossitem{Applet}{A small program that runs in a larger client context, often Java programs embedded in Web pages.}
\glossitem{AppsWG}{Applications Working Group}
\glossitem{Architecture}{The overarching design of a computer, network, or software system.}
\glossitem{Array}{A data structure for software elements where each element has a unique identifying index number.}
\glossitem{ASCII}{(American Standard Code for Information Interchange) Formally, an encoding of common alphanumeric symbols. Often used to mean a human readable representation with no ‘special’ characters or formatting.}
\glossitem{ASDM}{Archive Science data model : a data format for ALMA , EVLA data}
\glossitem{ASP}{(Active Server Pages) A technology that enables dynamic web pages using server -side scripting.}
\glossitem{Asynchronous services}{See Synchronous/Asynchronous services}
\glossitem{Astropy}{A community Python Library for Astronomy. Astropy
aims to enable the community to develop a robust ecosystem of affiliated
packages covering a broad range of needs for astronomical research, data
processing, and data analysis. See also PyVO}
\glossitem{Attributes}{1. In XML, characteristics of an element that are
specified within the $<  >$ brackets after the name of the element, e.g. . 2. In database management systems, the term attribute is sometimes used as a synonym for field (i.e. column).}
\glossitem{Bindings}{Associations between defined web interfaces and the services that provide them.}
\glossitem{Authentication}{In computer security, verification of the identity of a user and/or the user's eligibility to access a service.}
\glossitem{C\#}{An object-oriented programming language from Microsoft that is based on C++ with elements from Visual Basic and Java.}
\glossitem{C++}{An object-oriented version of the C programming language.}
\glossitem{Callbacks}{Routines that are automatically invoked when some event occurs or situation is encountered.}
\glossitem{Capability}{What a service can do - i.e. whether it offers a cone search service, or a TAP service, or a plain web interface, etc. A service can offer more than one capability.}
\glossitem{Carnivore}{Open source registry developed at Caltech, supporting publishing, searching, and harvesting. Primarily intended for use by data providers who want to set up their own registry.}
\glossitem{CC0}{A label, frequently provided when a usage license might be expected, that indicates the associated resource (dataset, software, etc.) has been explicitly dedicated into the worldwide public domain using the Creative Commons dedication tool. Resources dedicated into the worldwide public domain can be used without any restriction or legal requirement for attribution or citation. Attribution/citation is still ethically advisable.}
\glossitem{CC-BY}{A license, typically used with a version number as in "CC-BY-4.0" that requires attribution of the copyright holder but otherwise places no constraints on use or reused of the associated resource. This is the most permissive of the Creative Commons licenses.}
\glossitem{Certificate}{An electronic document that verifies the owner of a public key, issued by a certificate authority.}
\glossitem{CGI}{(Common Gateway Interface) A protocol that defines how data is passed to server applications using HTTP.}
\glossitem{Client}{A computer program or terminal that requests information or services from another computer (a server) on the network.}
\glossitem{Code stubs}{Code, usually generated by software tools, which defines the interfaces to some component but typically does not include any implementation of its functionality.}
\glossitem{Cone search}{See SCS.}
\glossitem{Container}{An element that acts as a parent and contains child elements.}
\glossitem{CORBA}{(Common Object Request Broker Architecture) An open, vendor-independent architecture and infrastructure that computer applications use to work together over networks.}
\glossitem{Crossmatch}{Find objects from two or more datasets that are near each other in the sky.}
\glossitem{CSP}{IVOA Standing Committee on Science Priorities}
\glossitem{Cyberinfrastructure}{A research environment in which advanced computational services are available to researchers through high-performance networks.}
\glossitem{Daemon}{A program or process that runs in the background unattended and may be invoked by another process to perform its function.}
\glossitem{DAL}{(Data Access Layer) The VO protocols that define how
VO applications access data resources. Data Access Layer Working
Group}
\glossitem{DALI}{The Data Access Layer Interface (DALI) defines resources, parameters, and responses common to all DAL services so that concrete DAL service specifications need not repeat these common elements.}
\glossitem{DALWG}{Data Access Layer Working Group}

\glossitem{Data model}{A formal description of how data may be structured and used.}
\glossitem{Database management system}{A collection of programs that enables storage, modification, and information extraction from a database. Also see RDBMS.}
\glossitem{Datalink}{DataLink is a data access protocol with the purpose to provide a mechanism to link resources found via one service to resources provided by other services.}
\glossitem{DataScope}{A web-based VO tool that finds information from many VO sources near a specified point in the sky.}
\glossitem{DCPIG}{Data Curation and Preservation Interest Group}
\glossitem{Distributed database}{A database where the underlying data is stored on multiple servers.}
\glossitem{Distributed computing}{Spreading the workload for processing tasks over multiple machines.}
\glossitem{DCOM}{(Distributed Common Object Model) A protocol that allows communication and manipulation of objects over a network connection.}
\glossitem{DHCP}{(Dynamic Host Configuration Protocol) A protocol that automatically manages IP addresses for a set of nodes in a network.}
\glossitem{DMWG}{Data Modelin Working Group}
\glossitem{DOI}{Digital Object Identifier - a permanent identifier supported by the DOI Foundation and issued through the authority of a DOI Foundation Registration Agency, like DataCite or CrossRef.}
\glossitem{DOM}{(Document Object Model) A W3C standard in which a structured document such as an XML file is viewed as a tree of elements.}
\glossitem{DPS}{The Division of Planetary Sciences of the American Astronomical Society (AAS). This might also refer to the annual meeting of the DPS, typically held someplace in North American during the northern Autumn and focusing on planetary science. Every 3 years or so the DPS and the EPSC hold a joint meeting somewhere in the world.}
\glossitem{EduIG}{Education Interest Group}
\glossitem{Element}{1. A single item in an array. 
2. In XML, a node in document. Each element starts with a and ends with.}
\glossitem{EPN-TAP}{A framework for using the Table Access Protocol (TAP) with the EPNCore metadata dictionary to publish solar system data to the IVOA.}
\glossitem{EPNCore}{A metadata dictionary that defines terms designed to facilitate discovery of solar system data services in the IVOA. It is part of the EPN-TAP recommendation.}
\glossitem{EPSC}{Europlanet Science Conference, an annual meeting typically held during northern Autumn and held in a different venue worldwide each year. Every three years or so, EPSC and DPS hold a joint meeting somewhere in the world. The focas of the meeting is planetary science and supporting disciplines (coding, archiving, etc.).}
\glossitem{Europlanet}{A reference to the Europlanet Society and any of its sponsored activities, including Research Initiatives (RIs).}
\glossitem{Exec}{IVOA Executive Committee}
\glossitem{Federation}{The dynamic combination of information from separate sources of information.}
\glossitem{FITS}{(Flexible Image Transport System) The IAU-approved standard format for astronomical data.}
\glossitem{FTP}{(File Transfer Protocol) A protocol used to exchange files over the internet.}
\glossitem{Footprint}{The region of the sky that has been observed by one or more telescopes.}
\glossitem{F\&P}{Fields and Particals; a reference to a type of data recording observations of fields and particles (other than dust).}
\glossitem{GPL}{(GNU Public License) A software license which allows for redistribution but requires both original and modified source code to be made available.}
\glossitem{Grid}{Massive distributed computing capabilities currently available on the Internet.}
\glossitem{Grid computing}{Applying the resources of many computers in a network to a single problem at the same time.}
\glossitem{GUI}{(Graphical User Interface) A graphics-based user interface that incorporates movable windows, icons and a mouse.}
\glossitem{GWSWG}{Grid and Web Services Working Group}
\glossitem{HiPS}{The Hierarchical Progressive Survey aims to  enable
dedicated  client/browser  tools  to  access  and  display  an
astronomical survey progressively, based on the principle that “the more
you zoom in on a particular area the more details show up”. It is based
on the healpix tessellation.}
\glossitem{HORIZONS, also "JPL Horizons"}{An ephemeris and cross-identification service for solar system bodies, in particular small bodies, hosted by the Jet Propulsion Laboratory (JPL) Solar System Division (SSD).}
\glossitem{HPC}{(High Performance Computing) Typically refers to supercomputers used in scientific research.}
\glossitem{HTML}{(Hypertext Markup Language) A standard document format used on most web pages which makes it easy for one document to refer to another.}
\glossitem{HTTP}{Communications protocol used to access most web pages.}
\glossitem{HTTPS}{A secure version of the HTTP protocol.}
\glossitem{HTTP GET}{An HTTP request where any parameters of the request are included in the URL itself.}
\glossitem{HTTP POST}{An HTTP request where data is sent to the server.}
\glossitem{IDL}{(Interactive Data Language) A popular data analysis programming language used by scientists.}
\glossitem{IMCCE}{Institut de Mécanique Céleste et de Calcul des Éphémérides; ephemeris and cross-identification services provided by Observatoire de Paris.}
\glossitem{Instance document}{An XML document that conforms to a schema.}
\glossitem{Interface}{A structured interaction between two entities, often a client and a server.}
\glossitem{Interoperability}{The ability of software and/or hardware on different machines from different vendors or sources to share data or collaborate without special effort on the part of the user.}
\glossitem{Interpreted language}{A programming language that is parsed and executed without needing any explicit compilation.}
\glossitem{IPDA}{International Planetary Data Alliance - an association of national planetary archives that have joined together to promote interoperability and open science worldwide. The IPDA has endorsed the PDS4 standard its recommend archive format for interoperability, and contributes to PDS4 design discussions through representative serving on the related working groups.}
\glossitem{IRAF}{(Image Reduction and Analysis Facility) A software system for astronomical data analysis including both tools, libraries and languages developed primarily at the National Optica Astronomy Observatories (NOAO).}
\glossitem{IVOA}{(International Virtual Observatory Alliance) An international collaboration formed in June 2002 to coordinate Virtual Observatory activities worldwide.}
\glossitem{IVOA identifiers}{An IVOA Identifier is a globally unique name for a resource within the Virtual Observatory. }
\glossitem{IVORN, IVOA identifier}{A standardized, unique ID used in a number of contexts within the VO.}
\glossitem{Java}{An object-oriented, platform-independent programming language, developed by Sun, and modeled after C++.}
\glossitem{JDBC}{(Java Database Connectivity) The standard Java interface for access to SQL -based DBMS ’s.}
\glossitem{KDDIG}{Knowledge Discovery Interest Group}
\glossitem{Markup language}{A language for annotating a document to enable each component to be appropriately formatted, displayed, or used.}
\glossitem{Metadata}{Information or labels that describe data.}
\glossitem{Middleware}{An intermediate level of computer software typically used to provide a common interface to heterogeneous lower level components.}
\glossitem{MIME}{(Multipurpose Internet Mail Extensions) The most common protocol for encoding, transmitting, and decoding non-text files via e-mail.}
\glossitem{Mirage}{VO-enabled tool for exploratory analysis and visualization of images and multi-dimensional numerical data.}
\glossitem{MOC}{The Multi-Order Coverage map is an IVOA standard to manipulate coverages in order to provide very fast union, intersection and equality operations between them. It is based on the healpix tesselation.}
\glossitem{Mosaic, mosaicking}{A virtual observation made by combining multiple observations at different positions resulting in a combination of images that abut and/or overlap into a single larger image.}
\glossitem{MPC}{The Minor Planet Center, located at the Center for Astrophysics at Harvard \& Smithsonian. The MPC is sanctioned by the IAU to assign identifications to small solar system bodies, and is the international coordination point for reporting observations of small bodies and generating orbit solutions.}
\glossitem{MySQL}{An open source SQL relational database management system (RDBMS).}
\glossitem{NAIF}{Navigation and Ancillary Information Facility, a node of the Planetary Data System that specializes in creating software and input files for calculating pointing, field of view, astrometric time, and so on for noth remote sensing missions and Earth-based observations.}
\glossitem{Name resolver}{A service that translates object names into astronomical coordinates.}
\glossitem{Namespace}{1. The set of names in a naming system.
2. In XML, a collection of names, identified by a URI reference, that are used in XML documents as element types and attribute names.}
\glossitem{NED}{(NASA Extragalactic Database) An astronomical resource containing extensive information on extragalactic objects. NED also provides a name resolver.}
\glossitem{NESSSI}{(NVO Extensible Scalable Secure Service Infrastructure) A VO web service that runs secure, asynchronous services on the Grid.}
\glossitem{OAI}{(Open Archive Initiative) A protocol that defines an interface for the sharing of metadata.}
\glossitem{OASIS}{(On-line Archive Science Information Services) A VO Tool to access and display astronomical image and catalog data.}
\glossitem{Object-oriented}{Programming based on the concept of an “object”, a data structure associated with specific routines that define the behavior of the object.}
\glossitem{ObsCoreDM}{The Observation Core components Data Model
describes the core metadata common to most data products distributed for
astronomical observations. It is the common basis that helps to search and discover datasets across various VO compatible archives via a customized TAP protocol: ObsTAP}
\glossitem{ObsLocTAP}{The Observation Locator Table Access Protocol (ObsLocTAP) defines a data model for scheduled observations and a method to run queries over compliant data, using several Virtual Observatory technologies.}
\glossitem{ObsTAP}{TAP interface to ObsCoreDM}
\glossitem{Ontology}{A formal description of the vocabulary used in a field, especially describing the relationships between various concepts within that subject.}
\glossitem{Open source}{Software, usually free, created by a development community where the source code is distributed as well as compiled code.}
\glossitem{OpenSkyQuery}{A VO Service that enables crossmatching of astronomical catalogs and selection of catalog subsets.}
\glossitem{Operating system}{The program framework in which all other programs run on a computer, e.g. Windows XP, MacOS? X, Linux, etc.}
\glossitem{Operations IG}{Operations Interest Group}
\glossitem{Parameters}{Inputs to or elements in a system which can be varied.}
\glossitem{Parse}{To separate into more easily understood parts.}
\glossitem{PDS}{The Planetary Data System, created by NASA to be the permanent archive for all NASA planetary science missions and related data.}
\glossitem{PDS3/PDS4}{Version 3 and 4, respectively, of the PDS data formatting and archiving standards. The PDS3 standard was superseded in about 2012 by the PDS4 standard, which is based on an information model and currently implemented as XML labels to describe the data structure and metadata.}
\glossitem{Peer-to-peer}{Communication between two or more computers where the protocol is symmetric between the participants so that each participant can make the same requests or give the same responses. This contrasts with client-server protocols}
\glossitem{Perl}{A programming language frequently used for web scripts and to process data passed via HTML forms.}
\glossitem{PHP}{(PHP [personal home page] Hypertext Preprocessor) A scripting language used to create dynamic Web pages.}
\glossitem{PhotDM}{The Photometry Data Model (PhotDM) standard describes photometry filters, photometric systems, magnitude systems, zero points and its interrelation with
the other IVOA data models through a simple data model.}
\glossitem{PLASTIC}{(PLatform for AStronomical Tool InterConnection? ) A protocol that allows collaboration between multiple processes running on the user’s desktop.}
\glossitem{Platform}{The combination of a computer’s operating system software and hardware.}
\glossitem{Portal}{A web site that serves as a starting point to other destinations or services on the web.}
\glossitem{PPI}{The Planetary Plasma Interactions node of the PDS, specializing in Fields and Particles data. The PPI node has implemented an EPN-TAP interface for part of its data holdings.}
\glossitem{Protocol}{A set of rules that define the interactions between two or more components.}
\glossitem{ProvenanceDM}{The Provenance Data Model describes how provenance information can be modeled, stored and exchanged within the astronomical community in a standardized way.}
\glossitem{Proxy}{A piece of software that acts on behalf of a user or another piece of software. An agent is a client proxy.}
\glossitem{Proxy certificate}{A certificate that is used in the place of another, typically with a limited lifetime.}
\glossitem{PSA}{The Planetary Science Archive of the European Space Agency. PSA is the ESA equivelent of NASA's PDS. Both use the PDS3/PDS4 data standards to format data.}
\glossitem{PyVO}{Astropy affiliated package providing access to remote
data and VO services using Python.}
\glossitem{Query}{To interrogate a collection of data such as records in a database.}
\glossitem{Radio IG}{Radio Interest Group}
\glossitem{RDBMS}{(Relational DataBase? Management System) A DBMS that uses represents data using a relational database.}
\glossitem{RDF}{(Resource Description Framework) A recommendation from the W3C for creating metadata structures that define data on the web.}
\glossitem{Registry}{The “yellow pages” for the VO. Collects and stores basic information about archives, data collections, databases, and other resources.}
\glossitem{Registry WG}{Resource Registry Working Group}
\glossitem{RegTAP}{The IVOA Registry Relational Schema defines an interface for searching this resource metadata based on the IVOA's TAP protocol.}
\glossitem{Relational database}{A database that stores data in a structure consisting of one or more tables (aka relations) of rows and columns, which may be interconnected.}
\glossitem{Resource metadata}{The metadata that describes services and data resources available in the VO.}
\glossitem{REST}{(Representational State Transfer) An approach to web services that uses the standard HTTP GET and POST protocols.}
\glossitem{RMI}{(Remote Method Invocation) A Java protocol for distributed computing.}
\glossitem{ROME}{(Request-Object Management Environment) A VO tool to manage the execution of a task that requires many subtasks.}
\glossitem{RPC}{(Remote Procedure Call) Protocols for distributed computing where the interaction is represented as the client computer invoking discrete services/calls on the server.}
\glossitem{RSS}{(Rich Site Summary). An XML format for sharing content among different Web sites such as news items.}
\glossitem{Ruby}{An object-oriented programming language.}
\glossitem{SAMP}{Messaging protocol that allows applications to exchange messages and data between each other through a central hub, which is a tiny application that runs on the user's desktop. For example, you might have retrieved a catalogue from a cone search service using Topcat, but also have an image tool such as Aladin or DS9 running, looking at an image of the same piece of sky. Topcat can then send the table to the image tool, which can overplot the objects on the image.}
\glossitem{SAX}{(Simple API for XML) A standardized interface for parsing XML documents using callbacks.}
\glossitem{SBN}{The Small Bodies Node of the PDS. The SBN, which contains the MPC and also manages the IAWN (International Asteroid Warning Network) website, is primarily responsible for archiving data related to asteroids, comets, dust (interplanetary and associated with small bodies), meteors/meteorites, interplanetary objects, small satellites, and other small solar system objects not within the purview of other PDS nodes.}
\glossitem{Schema}{1. A description of the structure and rules an XML document must satisfy. 
2. In SQL, a description of the tables and columns in the database.}
\glossitem{SCS}{A VO protocol that requests information near a specified location in the sky.}
\glossitem{SCSP}{Standing Committee on Standards and Processes}
\glossitem{Script}{A simple program usually written in an interpreted language.}
\glossitem{Semantics}{The expression of the meaning of symbols or names. In the VO, the actual scientific meaning of data and services.}
\glossitem{Semantics WG}{Semantics Working Group}
\glossitem{Serialization}{The process of converting an object into a format that can be stored or transmitted across a network.}
\glossitem{Server}{A computer system in a network whose services may be invoked by one or more clients.}
\glossitem{Service}{Something on the internet which will actively do
somethin, as opposed to being a passive repository of information. For example, an image service may have a large atlas of images, but also offers a way of submitting a query to get back a cut-out image from a particular piece of sky.}
\glossitem{Servlet}{A program, typically Java, that runs on a web server in response to a web request.}
\glossitem{SESAME}{A web service interface to the SIMBAD name resolver.}
\glossitem{Sexagesimal}{Numeral system with number 60 as the base.}
\glossitem{SExtractor}{(Source Extractor) A tool that detects sources in astronomical images.}
\glossitem{SIA, SIAP}{(Simple Image Access Protocol) A VO protocol that supports queries for images available in a given data collection near a given position on the sky.}
\glossitem{SIMBAD}{(Set of Identifications, Measurements and Bibliography for Astronomical Data) An astronomical database provides extensive information on both galactic and extragalactic objects. SIMBAD also provides a name resolver.}
\glossitem{SimDM}{(Simulation Data Model) describes numerical computer simulations of astrophysical systems.}
\glossitem{Simple Spectral Access}{See SSAP.}
\glossitem{Single Sign On}{Many datasets in the VO are public, but many have some proprietary restrictions, that need a user to specify who they are, and for the service to work out if they are allowed access. The idea behind single sign-on is that you should only have to sign-on once per session, and have your credentials passed around as necessary. The technical infrastructure for this is agreed but not yet fully implemented.}
\glossitem{SkyNode}{A VO protocol (and the services that implement it) that provides an ADQL interface to astronomical databases.}
\glossitem{SkyPortal}{A web site that supports translation of a user ADQL query into queries of one or more SkyNodes.}
\glossitem{SkyServer}{Web service that presents data from the Sloan Digital Sky Survey.}
\glossitem{SkyView}{Web site and VO-enabled distributable tool that generates images from survey data.}
\glossitem{SLAP}{The Simple  Line  Access  Protocol  (SLAP)  is  an
IVOA Data  Access  protocol which defines a protocol for retrieving
spectral lines coming from various SpectralLine  Data  Collections
through  a  uniform  interface  within  the  VO.}
\glossitem{SMTP}{(Simple Mail Transfer Protocol) A protocol used to send and receive email.}
\glossitem{SOA}{(Service Oriented Architecture) An approach to distributed computing that focuses on services that communicate with each other.}
\glossitem{SOAP}{(Simple Object Access Protocol) A protocol for invoking remote services by exchanging XML -based messages.}
\glossitem{SODA}{The SODA web service interface defines a RESTful web service for performing server-side operations and processing on data before transfer.}
\glossitem{Socket}{The low-level software element that makes a connection to the network. Normally a client connects to a socket on a server.}
\glossitem{Solar IG}{Solar Systems Interest Group}
\glossitem{Source code}{The version of a program normally written or edited by a programmer and either compiled into an executable program, or run directly using an interpreter (see interpreted languag).}
\glossitem{SPICE}{An acronym related to the various elements of a remote sensing mission used to compute observational geometry for remote sensing missions. It has come to be a short-hand reference for the geometry engines and software produced by the NAIF node of the PDS.}
\glossitem{SQL}{(Structured Query Language) The standard language used to communicate with RDBMS s}
\glossitem{SQL Server}{Microsoft’s RDBMS software.}
\glossitem{SRB}{(Storage Resource Broker) Middleware developed at the San Diego Supercomputing Center that provides standardized access to a number of very large data resources.}
\glossitem{SSAP}{(Simple Spectral Access Protocol) A protocol that returns a set of spectra in a specified region of the sky. Similar to SIA but has many more options.}
\glossitem{SSDL}{(SOAP Service Description Language) SSDL is a SOAP -centric description language for web services that enables protocol -based integration.}
\glossitem{SSL}{(Secure Sockets Layer) A protocol for managing the security of a message transmission over the Internet.}
\glossitem{Standalone application}{A computer program capable of operating without external resources.}
\glossitem{STC}{(Space-Time Coordinates) An IVOA standard for describing a region or position in both space and time.}
\glossitem{STILTS}{(STIL Tool Set) A set of VO tools for processing of tabular data based on the UK Starlink Tables Infrastructure.}
\glossitem{ST-MOC}{Space-Time Multi-Order Coverage map - a HIPS-based mapping that includes a time serialization. Time is generally required for planetary applications.}
\glossitem{Synchronous/Asynchronous services}{A synchronous service is one where the user needs to stay connected to a remote service and interact with it in real time, whereas with an asynchronous service the user specifies a job to be done, disconnects and collects the result later. For this to work, the IVOA has agreed a standardised way of describing a job, and the current state of a job.}
\glossitem{TAP}{Table Access Protocol. TAP services provide query-driven access to astronomical tables and databases. For example, whereas a simple cone-search allows you to search only by sky position, and returns a fixed set of columns, a TAP service allows to make searches along the lines of "give me all the records with B-V>2.0 and give me just columns B, D, F, and G". Queries need to be formulated in the standard ADQL, but often the tool you are using will construct this for you.}
\glossitem{Taplint}{A TAP service validator than runs a series of tests on a TAP service to check for compliant behaviour.}
\glossitem{TCG}{IVOA Technical Coordination Group}
\glossitem{TDIG}{Time Domain Interest Group}
\glossitem{Theory IG}{Theory Interest Group}
\glossitem{TLA}{(Three Letter Acronym) A tribute to the use of acronyms in the computer field.}
\glossitem{Token}{An item in a string of text that can be separated out by a parser, such as a single word in a sentence or a number in a comma-delimited list.}
\glossitem{TOMCAT}{An HTTP server that can run Java servlets.}
\glossitem{TOPCAT}{(Tool for OPerations on Catalogs And Tables) An interactive graphical viewer and editor for tabular data, designed for but not limited to astronomical tables.}
\glossitem{Treeview}{A VO-enabled viewer for hierarchical file structures.}
\glossitem{UCD}{(Unified Content Descriptors) A formal vocabulary for astronomical data that is controlled by the IVOA.}
\glossitem{URI}{(Uniform Resource Identifier) An address standard for a resource available on the Internet.}
\glossitem{URL}{(Uniform Resource Locator) The global address of documents and other resources on the World Wide Web. The first part of the address specifies the protocol to be used when accessing the resource, the remainder describes its network location.}
\glossitem{UWS}{The Universal Worker Service (UWS) pattern defines how to manage asynchronous execution of jobs on a service.}
\glossitem{Validator}{A tool that checks some element of a system for conformance to a standard.}
\glossitem{VESPA}{Virtual European Solar and Planetary Access, a Europlanet Research Infrastructure (RI) project. There is a VESPA portal associated with the project that provides access to IVOA services for solar and planetary datasets.}
\glossitem{Virtual data}{Data product that is dynamically generated when needed.}
\glossitem{Vizier}{A very large collection of astronomical catalogues and tables stored at CDS, Strasbourg. As well as well known catalogues, it includes many tables published as part of astronomical papers. Vizier catalogues and tables can be accessed through the CDS web pages, or through several different VO tools.}
\glossitem{VOClient}{A software suite callable from many languages which implements data access in the VO.}
\glossitem{VODataService}{VODataService refers to an XML encoding standard for a specialized extension of the IVOA Resource Metadata that is useful for describing data collections and the services that access them.}
\glossitem{VOEvent}{A VO standard for representing, transmitting, publishing and archiving the discovery of a transient celestial event.}
\glossitem{VOEventNet}{A peer-to-peer cyberinfrastructure to enable rapid and federated observations of the dynamic night sky.}
\glossitem{VOPlot}{A VO Tool for visualizing astronomical data from VOTable sources.}
\glossitem{VOResource}{VO Rsource describes an XML encoding standard for IVOA Resource Metadata.}
\glossitem{VOSI}{The VO Support Interface describes the minimum interface that a web service requires to participate in the IVOA.}
\glossitem{VOSpace}{A distributed storage concept for the VO.}
\glossitem{VOStat}{A VO and web-enabled statistics package.}
\glossitem{VOTable}{An XML -based encoding scheme for astronomical tables and catalogs, established by the IVOA in order to provide an unambiguous way to transmit tables between computer programs}
\glossitem{UCD}{Unified Content Descriptor. A standard vocabulary for describing astronomical data quantities. It does not specify the name of a quantity, or its units, but rather what <i>type</i> of quantity it is. For example, a column in table might have the name "T-kin" and the UCD "phys.temperature" which states that it represents a temperature, but does not imply a specific unit.}
\glossitem{Units}{IVOA standardised strings to specify the units of a quantity. At the time of writing this is in the final stages of agreement by the IVOA.}
\glossitem{W3C}{(World Wide Web Consortium) An international consortium where member organizations, a full-time staff, and the public work together to develop web standards.}
\glossitem{WCS}{(World Coordinate System) A detailed specification of the conversion between coordinates within a file and physical coordinates, especially between pixel and celestial coordinates in an image.}
\glossitem{WCS Fixer}{A VO web service that corrects the WCS information in a given FITS image.}
\glossitem{Web service}{Software available over the web using a standardized XML messaging system.}
\glossitem{WESIX}{A VO web service to the standard astronomical image analysis package SExtractor together with a crossmatching service.}
\glossitem{Wget}{Free software package for retrieving files using HTTP, HTTPS and FTP.}
\glossitem{Workflow}{A sequence or network of tasks and associated information needed to pass from one task to another to accomplish some goal.}
\glossitem{WS}{see web service}
\glossitem{WSDL}{(Web Services Description Language) An XML document that describes and locates a web service.}
\glossitem{XMatch}{see crossmatch}
\glossitem{XML}{(eXtensible Markup Language) A markup language that provides a file format for representing data.}
\glossitem{XML-RPC}{(XML-Remote Procedure Call) A web service protocol that utilizes XML technology to implement an RPC protocol.}
\glossitem{XOP}{(XML Binary Optimized Packaging) A standard specifying how binary data should be represented in XML.}
\glossitem{XPath}{(XML Path Language) a language that describes how to locate specific content within an XML document.}
\glossitem{XQuery}{A standard language for querying XML data.}
\glossitem{XSL}{(eXtensible Stylesheet Language) A standard for describing how to transform XML documents into other documents in either XML or other formats.}
\glossitem{XSLT}{(eXtensible Stylesheet Language Transformations) A conversion tool that implements XSL.}
\glossitem{YAML}{(YAML Ain’t Markup Language) A data serialization
language based on XML and other languages.}

\end{description}

